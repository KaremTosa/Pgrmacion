\documentclass{article}

\usepackage[spanish]{babel}
\usepackage{amsmath}
\usepackage[utf8]{inputenc}

\title{El método simplex}
\author{Karem Torres Salinas}

\begin{document}
\maketitle

\section{Introducción}
\label{sec:introduccion}
El método simplex es un algoritmo para resolver problemas de
programación lineal. Fue inventado por George Dantzing en el año 1947.

\section{Ejemplo}
\label{sec:ejemplo}

Ilustraremos la aplicación del método simplex con un ejemplo.

\begin{equation*}
 \begin{aligned}
\text{Maximizar} \quad & 2x_{1}+x_{2}\\
\text{sujeto a} \quad &
  \begin{aligned}
   x_{1}-x_{2} &\leq 2\\
   -2x_{1}+x_{2} &\leq 2\\
   3x_{1}+4x_{2} &\leq 12\\
   x_{1}+x_{2} &\geq 1\\
    x_{1},x_{2} &\geq 0
  \end{aligned}
\end{aligned}
\end{equation*}

Como una de las desigualdades aparecen las variables de lado izquierdo
de un símbolo mayor o igual, entonces multiplicamos ambos miembros de
la desigualdad por $-1$ para obtener la forma estándar.


\begin{equation*}
 \begin{aligned}
\text{Maximizar} \quad & 2x_{1}+x_{2}\\
\text{sujeto a} \quad &
  \begin{aligned}
   x_{1}-x_{2} &\leq 2\\
   -2x_{1}+x_{2} &\leq 2\\
   3x_{1}+4x_{2} &\leq 12\\
  - x_{1}-x_{2} &\leq -1\\
    x_{1},x_{2} &\geq 0
  \end{aligned}
\end{aligned}
\end{equation*}

Para obtener la forma simplex, añadimos una variable de holgura por
cada desigualdad.

\begin{equation*}
 \begin{aligned}
\text{Maximizar} \quad & 2x_{1}+x_{2}\\
\text{sujeto a} \quad &
  \begin{aligned}
   x_{1}-x_{2}+x_{3}              &  =2\\
   -2x_{1}+x_{2}+x_{4} &=2\\
   3x_{1}+4x_{2}+x_{5} &=    12\\
  - x_{1}-x_{2}+x_{6} &=    -1\\
    x_{1},x_{2},x_{3},x_{4},x_{5},x_{6} &\geq 0
  \end{aligned}
\end{aligned}
\end{equation*}

A continuación obtenemos un \emph{tablero simplex} despejando las
variables de holgura.

\begin{equation*}
 \begin{aligned}
\text{Maximizar} \quad & 2x_{1}+x_{2}\\
\text{sujeto a} \quad &
\begin{aligned}
  x_{3}&= 2+-x_{1 }+x_{2} \\
  x_{4}&=2-x_{1}-x_{2}-x_{4}\\
  x_{5}&=12-3_{1}-4x_{2}\\
  x_{6}=&-1+x_{1}+x_{2}\\
  \hline
  z&=\phantom{-1}+2x_{1}+x_{2}
  \end{aligned}
\end{aligned}
\end{equation*}



\end{document}
