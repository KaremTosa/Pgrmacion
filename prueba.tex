\documentclass{article}

\usepackage[utf8]{inputenc}
\usepackage{amsmath}
\usepackage[spanish]{babel}
\title{Apuntes de programación lineal}

\author{Karem Torres Salinas}



\begin{document}


\maketitle
\tableofcontents
\label{sec:introduccion}
\section{Introduccion}



La forma estándar de un problema de programación lineal es: Dados una
matriz $A$ y vectores $b,c$, maximizar $c^Tx$ sujeto a $Ax\leq b$.

\subsection{Ejercicios}
Ejercicio 1:
Una compañia produce frutas mezcladas tiene en almacén 10,000 kilos de
peras, 12,000 kilos de duraznos y 8,000 kilos de cerezas. La compañia
produce tres mezclas de frutas que venden en latas de un kilo. La
primera combinación contiene la mitad de peras y la mitad de
duraznos. La segunda combinación contiene la mitad de cada fruta . La
tercera combinación tiene la mitad de duraznos y la mitad de
cerezas. Los ingresos por lata vendida por cada combinación son de 3,4
y 5 pesos respectivamente. Platea el problema de encontrar la
producción que da la ganancia máxima como un problema de programación
lineal. Escribe el problema en forma estándar y en forma simplex.


\subsection{Tabla}

\begin{tabular}{|c|c|c|}
  \hline
  &A&B\\
  \hline
  Maquina1 &1&2\\
  Maquina2 &1&1 \\
  \hline
\end{tabular}


\subsection{Matriz}

\begin{equation*}
  \label{eq:1}
 A= \begin{pmatrix}
    1&2&3\\
    4&4&6\\
    7&8&9\\
  \end{pmatrix}
\end{equation*}



\end{document}